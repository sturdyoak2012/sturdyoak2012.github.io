\documentclass[12pt]{article}
\usepackage{jacob}
\usepackage[left=1.5in,top=1.5in,bottom=1.5in,right=1.5in]{geometry}
\newcommand{\diff}{\,\textit{d}}
\title{}
\date{}
\author{Jacob H. Nie}

\begin{document}

\section*{Problem 13}

Let $\mathbf{m}_1$ denote the magnetic moment of the fixed dipole.  Let $\mathbf{m}_2$ denote the magnetic moment of the dipole that will move.  Let $ \mathbf{r}$ be the vector that points from $\mathbf{m}_1$ to $\mathbf{m}_2$.  Then the force between the two dipoles is given by
\begin{align}
	\mathbf{F} &= \frac{3\mu_0}{4\pi r^{5}} \Big[ (\mathbf{m}_1 \cdot \mathbf{r})\mathbf{m}_2 + (\mathbf{m}_2 \cdot \mathbf{r})\mathbf{m}_1 + (\mathbf{m}_1\cdot \mathbf{m}_2)\mathbf{r} - \frac{5 (\mathbf{m}_1 \cdot \mathbf{r})(\mathbf{m}_2 \cdot \mathbf{r})}{r^2}\mathbf{r} \Big]
\end{align}
which I found on Wikipedia.  

Let the magnitude of each magnetic moment be $m$.  In the first configuration (a), the following relations are true: $\mathbf{m}_1 \cdot \mathbf{r} = 0,$ $\mathbf{m}_2 \cdot \mathbf{r} = 0,$ $\mathbf{m_1}\cdot \mathbf{m}_2 = -m^2$.  Thus,
\begin{equation}
	\mathbf{F} = -\frac{3\mu_0 m^2}{4\pi r^5} \mathbf{r}.
\end{equation}

In the second configuration (b), the following relations are true: $\mathbf{m_1}\cdot \mathbf{r} = mr$, $\mathbf{m_2}\cdot \mathbf{r} = mr$, $\mathbf{m_1}\cdot \mathbf{m_2} = m^2$, and we will also define $\hat{\mathbf{r}}$ as the unit vector pointing in the $\mathbf{r}$ direction.  We will find that
\begin{equation}
	\mathbf{F} = - \frac{3 \mu_0 m^2}{2\pi r^5} \mathbf{r}.
\end{equation}

(2) and (3) are very similar, so we should be able to compute the elapsed times in very similar ways.  First, let the mass of the dipole be $w.$  Then define, for ease of notation,
\[
	\gamma \equiv \frac{3 \mu_0 m^2}{4 \pi w}.
\]
(2) becomes:
\begin{equation}
	\ddot{r} = -\gamma r^{-4}.
\end{equation}
We use the trick $a\ \diff x = v \ \diff v$ to write:
\begin{align*}
	\int_{0}^{\dot{r}} \dot{r}' \ \diff \dot{r} ' &= \int _{r_0}^{r} \ddot{r} \ \diff r' \\
						      &= \int_{r_0} ^{r} - \gamma r'^{-4} \ \diff r' \\
						      &= \left[ \frac{1}{3}\gamma r^{-3} \right]_{r_0}^r,
\end{align*}
which implies that
\begin{equation}
	\dot{r} = -\left[ \frac{2}{3}\gamma\left( \frac{1}{r^3} - \frac{1}{r_0^3} \right) \right]^{1/2},
\end{equation}
where we have taken the negative value because we know that $r$ is decreasing.  This is a separable differential equation that allows us to write:
\begin{align}
	t &= \int_0^{r_0} \left[ \frac{2}{3}\gamma\left( \frac{1}{r^3} - \frac{1}{r_0^3} \right) \right]^{-1/2} \ \diff r \nonumber \\
	  &= \left( \frac{3}{2\gamma} \right)^{1/2} \int_0^{r_0} \frac{r^{3/2}}{\left[1-(r/r_0)^3\right]^{1/2}} \ \diff r.
\end{align}
(6) is rather difficult to integrate.  I used Mathematica.  The answer was
\begin{align}
	t &= \left( \frac{3}{2\gamma} \right)^{1/2} \frac{\Gamma(5/6)}{\Gamma(1/3)} \pi^{1/2} r_0^{5/2} \nonumber \\
	  &= \left( \frac{2\pi^2 w}{\mu_0 m^2} \right) ^{1/2} \frac{\Gamma(5/6)}{\Gamma(1/3)} r_0^{5/2}.
\end{align}

In part (b), the answer is similar, except $\gamma \to 2\gamma$.  This yields
\begin{equation}
	t = \left( \frac{\pi^2 w}{\mu_0 m^2} \right) ^{1/2} \frac{\Gamma(5/6)}{\Gamma(1/3)} r_0^{5/2}.
\end{equation}
\vspace{1cm}
\hfill Contestant: Jacob H. Nie







\end{document}
