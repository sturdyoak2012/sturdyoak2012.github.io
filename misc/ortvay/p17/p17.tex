\documentclass[12pt]{article}
\usepackage{jacob}
\usepackage[left=1.5in,top=1.5in,bottom=1.5in,right=1.5in]{geometry}

\title{}
\date{}
\author{Jacob H. Nie}

\begin{document}

\section*{Problem 17}

Let $\phi$ be the rotational angle of the diffraction grating.  Let $\theta$ be the angle with which the light beam in question comes out of the diffraction grating, as shown in the figure.  The ``scattering'' angle is then $\theta - \phi$. 
\begin{figure}[ht]
	\centering
	\incfig{figure}
	\caption{In the figure, all angles as shown as positive.  They can become negative, since they are directed angles.}
\end{figure}

The light beam comes into the grating (below the first dashed line) in phase, and it must come out in phase as well.  Thus, the difference in distance travelled between rays A and B must be an integral multiple of the wavelength $\lambda$.  Let us examine within each region:

\begin{itemize}
	\item In region I, ray A is shorter by $d\sin \phi$. 
	\item In region II, ray A is longer by $d\sin\theta$.
\end{itemize}

Therefore, we conclude:
\begin{equation}
	m \lambda = d\sin \theta - d\sin \phi.
\end{equation}
$m$ must be an integer.  We have additional knowledge about $\phi$ from the problem statement: $\phi$ is the same angle as the $n$-th order diffracted beam in the original setup.  That means 
\[
	\sin \phi = \frac{n\lambda }{d},
\] 
where $n$ is some positive integer that uniquely determines $\phi$.  (Without loss of generality, we assume that both $\phi$ and $n$ are positive.  The negative scenario just flips the setup and results in an identical final answer.)

This allows us to rewrite (1) as:
\begin{equation}
	\sin \theta = \frac{(n+m)\lambda }{d}.
\end{equation}
Thus, the final direction of the light beam is
\begin{equation}
	\theta - \phi = \arcsin\left[ \frac{(n+m)\lambda }{d} \right] - \arcsin\left( \frac{n\lambda }{d} \right).
\end{equation}

\vspace{1cm}

\noindent (c) The non-existence of certain diffracted beams is due to the bound $-1 \leq \sin \theta \leq 1$.  The question essentially asks for conditions such that $\theta$, as given by (2), exists for only $m=0$, one of $m=\pm 1$, and nothing else.  

Now, since we assumed $n$ to be positive, if $\theta$ exists for $m = 1$, then it must certainly also exist for $m = -1$.  Since only one can exist, we seek parameters such that $\theta$ does not exist for $m = 1$.  This will occur if $\sin \theta > 1$.  
\begin{equation}
	\frac{(n+1)\lambda }{d} > 1.
\end{equation}
However, there is also the additional condition on $n$ because if $n$ is too large, $\phi$ will not exist.  That is,
\begin{equation}
	\frac{n\lambda }{d} < 1.
\end{equation}
Therefore, we conclude that
\begin{equation}
\frac{d}{\lambda} - 1 < n < \frac{d}{\lambda}.	
\end{equation}
This condition suffices for $d > \lambda$.  However, if that is not the case, then we must also remember that $\theta$ must exist for $m = -1$.  That condition is
\begin{align}
	\frac{(n-1)\lambda}{d} > -1 \nonumber \\
	n > 1 - \frac{d}{\lambda}.
\end{align}
(7) is the lower bound for $n$ when $d < \lambda$.  

\vspace{1cm}

\noindent (d) If the rotation angle is a more general $\phi$, then our condition for $\theta$ becomes
\begin{equation}
	\sin \theta = \sin \phi + \frac{m\lambda}{d},
\end{equation}
for integer $m$.  If the condition as described in (c) is to be fulfilled, then we can argue along very similar lines.  

$\theta$ cannot exist for $m = 1$.  That is,
\begin{align}
	\frac{\lambda}{d} + \sin \phi > 1 \nonumber \\
	\lambda > d - d\sin \phi.
\end{align}
On the other hand,  $\theta$ must exist for $m = -1$.  That is,
\begin{align}
	-\frac{\lambda}{d} + \sin \phi &> - 1 \nonumber \\
	\lambda < d + d\sin\phi.
\end{align}
Thus, the ratio of maximum to minimum $\lambda$ is
\begin{equation}
	\frac{\lambda_{\text{max}}}{\lambda_{\text{min}}} = \frac{1 + \sin\phi}{1 - \sin\phi}.
\end{equation}


\vspace{1cm}

\hfill Contestant: Jacob H. Nie


\end{document}
