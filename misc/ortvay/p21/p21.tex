\documentclass[12pt]{article}
\usepackage{jacob}
\usepackage[left=1.5in,top=1.5in,bottom=1.5in,right=1.5in]{geometry}

\title{}
\date{}
\author{Jacob H. Nie}

\begin{document}

\section*{Problem 21}

First, we will go through the preliminary steps covered in most textbooks, including calculating the Fermi energy and deriving the density of states.

Then, we will answer the question of the problem.

\subsection*{Preliminaries}
Consider the Fermi electron gas inside a solid conducting cube of length $L$ at zero temperature.  The energies of each energy level is given by 
\begin{equation} \label{eq:eeig} 
	\epsilon_i = \frac{\hbar^2 \pi^2}{2 m L^2}(n_x^2 + n_y^2 + n_z^2) \equiv \frac{\hbar^2 \pi^2}{2 m L^2}n^2. 
\end{equation}
The Fermi-Dirac distribution
\begin{equation}
	f(\epsilon_i) = \left[ \exp\left( \frac{\epsilon_i - \mu}{kT} \right) + 1 \right]^{-1} 
\end{equation}
gives the expected occupancy of each orbital $i$. At zero temperature, this is
 \[
	 f(\epsilon_i) = \begin{cases} 1 & \qquad \epsilon_i < \epsilon_F \\ 0 & \qquad \epsilon_i > \epsilon_F \end{cases}
\]
where $\epsilon_F$ is the Fermi energy, the chemical potential $\mu$ at zero temperature.  The Fermi energy is the energy of the highest occupied orbital of the degenerate electron gas at zero temperature.  

\subsubsection*{Fermi energy}
Note that (\ref{eq:eeig}) implies that the allowed energies can be treated as the discrete points in the positive octant of a sphere formed in $\mathbf{n}$ space.  At zero temperature, the Fermi energy $\epsilon_F$ is the maximum allowed energy that is occupied.  Let us define $n_{\text{max}}$ as the $n$ that is associated with this orbital.  The total number of occupied orbitals $N$ will be the volume of this sphere in the positive octant times 2, due to the two spin states of the electron.  Thus,
\begin{align}
	N &= 2\left( \frac{4\pi}{3} \right)\left( \frac{1}{8} \right) n_{\text{max}}^{3} \nonumber \\
	  &= \frac{1}{3}\pi n_{\text{max}}^3.
\end{align}
Now 
\[
	n_{\text{max}} = \left( \frac{2 m L^2 \epsilon_F}{\hbar^2 \pi^2} \right)^{1/2},
\]
so 
\[
	N = \frac{\pi}{3}\left( \frac{2m \epsilon_F L^2}{\hbar^2 \pi^2} \right)^{3/2}
\]
which means
\begin{equation}
	\epsilon_F = \frac{\hbar^2 \pi^2}{2 m}\left( \frac{3N}{\pi V} \right)^{2/3}.
\end{equation}


\subsubsection*{Density of states}

Let us now calculate the density of states.  Note that, according to (\ref{eq:eeig}), the allowed energies can be treated as the discrete points in the positive octant of a sphere formed in $\mathbf{n}$ space.  We can express the number of orbitals between $n$ and $n + \diff n$ as 
\begin{equation}
	\frac{1}{2}\gamma \pi n^2 \diff n = \pi n^2 \diff n,
\end{equation}
where $\gamma$ is the degeneracy of each orbital.  Because electrons are spin-$1/2$, $\gamma = 2$.  Changing variables to $\epsilon$ by making use of (\ref{eq:eeig}), we can find the number of orbitals between $\epsilon$ and $\epsilon + \diff \epsilon$, 
\[
	\frac{\pi}{2}\left( \frac{2mL^2}{\hbar^2 \pi^2} \right)^{3/2} \epsilon^{1/2} \diff \epsilon.
\]
Expressing this as a distribution $g(\epsilon)$,
\begin{equation} \label{eq:dos}
	g(\epsilon) = \frac{\pi}{2}\left( \frac{2mL^2}{\hbar^2 \pi^2} \right)^{3/2} \epsilon^{1/2}.
\end{equation}
This can also be expressed in terms of the Fermi energy:
\begin{equation} \label{eq:dos-ef}
	g(\epsilon) = \frac{3N}{2\epsilon_F^{3/2}} \epsilon^{1/2}.
\end{equation}
\begin{remark}
	Although $g(\epsilon)$ contains an $N$ in (\ref{eq:dos-ef}), it is not dependent on $N$ because the $\epsilon_F$ cancels out this $N$ term. This is clearly seen in (\ref{eq:dos}).
\end{remark}

\subsection*{Specific Heat at Constant $\mu$ }
Typically, when the number of electrons is kept constant, $\mu$ will decrease with increasing temperature.  This leads to some complications in the calculation of specific heat.  Here, the problem asks us to consider the case where $\mu$ is held constant.  So the electron gas is being kept in electronic equilibrium with some reservoir---in other words, grounding.

For ease of notation, let 
\[
	g_0 \equiv \frac{\pi}{2} \left( \frac{2mL^2}{\hbar^2 \pi^2} \right)^{3/2},
\]
such that
\[
	g(\epsilon) = g_0 \epsilon^{1/2}.
\]
The energy of the gas is therefore 
\begin{align}
	U &= \int_{0}^{\infty}\epsilon g(\epsilon) f(\epsilon) \diff \epsilon \\
	  &= g_0 \int_0^{\infty} \epsilon^{3/2} \left[ 1+\exp\left( \frac{\epsilon - \mu}{kT} \right) \right]^{-1} \diff \epsilon. \label{eq:int-1}
\end{align}
Integrating by parts, (\ref{eq:int-1}) becomes 
\begin{equation*}
	U = g_0 \left\{ \frac{2}{5}\epsilon^{5/2} \left[ 1+\exp\left( \frac{\epsilon - \mu}{kT} \right) \right]^{-1} \Bigg|_0^{\infty} + \frac{2}{5kT}\int_0^{\infty} \epsilon^{5/2} \frac{\exp\left( \frac{\epsilon - \mu}{kT} \right)}{\left[ 1+\exp\left( \frac{\epsilon - \mu}{kT} \right) \right]^{2}} \diff \epsilon \right\}.
\end{equation*}
The first term vanishes, so this is 
\[
	U = \frac{2g_0}{5}\cdot \frac{1}{kT} \int_0^{\infty}  \epsilon^{5/2} \frac{\exp\left( \frac{\epsilon - \mu}{kT} \right)}{\left[ 1+\exp\left( \frac{\epsilon - \mu}{kT} \right) \right]^{2}} \diff \epsilon. 
\]
Substituting $x = (\epsilon - \mu)/kT$, this is 
\[
	U = \frac{2g_0}{5} \int_{-\mu/kT}^{\infty} \epsilon^{5/2} \frac{e^x}{(1+e^x)^2} \diff x.
\]
The integrand is very sharp at $x = 0$, and we can also assume $\mu \gg kT$, so it is acceptable to replace the lower limit with $-\infty$.  Additionally, we Taylor expand $\epsilon^{5/2}$ around $\mu$.  We can make this approximation because the second term of the integrand is so sharply peaked around $x = 0$ or $\epsilon = \mu$.  This expansion is
\begin{equation}
	\epsilon^{5/2} = \mu^{5/2} + \frac{5}{2}kT\mu^{3/2}x + \frac{15}{8}k^2 T^2 \mu^{1/2} x^2 + \cdots.
\end{equation}
Thus, we can split our integral into 
\begin{align}
	U &= \frac{2g_0}{5}\int_{-\infty}^{\infty} \Bigg [ \mu^{5/2}\frac{e^x}{(1+e^x)^2}  + \frac{5}{2}kT\mu^{3/2} \frac{xe^x}{(1+e^x)^2} \nonumber \\
	  &\qquad \qquad \frac{15}{8} k^2 T^2 \mu^{1/2} \frac{x^2 e^x}{(1+e^x)^2} + \cdots \Bigg ] \diff x. 
\end{align}
Using the three integrals
\begin{align*}
	\int_{-\infty}^{\infty} \frac{e^x}{(1+e^x)^2} &= 1 \\
	\int_{-\infty}^{\infty} \frac{xe^x}{(1+e^x)^2} &= 0 \\
	\int_{-\infty}^{\infty} \frac{x^2e^x}{(1+e^x)^2} &= \frac{\pi^2}{3},
\end{align*}
we determine that 
\begin{align}
	U &= \frac{2g_0}{5}\left( \mu^{5/2} + \frac{5\pi^2 k^2 T^2}{8}\mu^{1/2} \right)\nonumber \\
	  &= \frac{\pi}{5} \left( \frac{2mL^2}{\hbar^2 \pi^2} \right)^{3/2} \mu^{5/2} + \left( \frac{m}{2\hbar^2} \right)^{3/2} Vk^2 T^2 \mu^{1/2}.
\end{align}
Therefore, the specific heat is
\begin{equation}
	\frac{\partial U}{\partial T} = \left( \frac{m^3\mu}{2} \right)^{1/2} \frac{Vk^2T}{\hbar^3}.
\end{equation}

\vspace{1cm}

\hfill Contestant: Jacob H. Nie


\end{document}




















