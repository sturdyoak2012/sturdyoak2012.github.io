\documentclass[12pt]{article}
\usepackage{jacob}
\usepackage[left=1.5in, right=1.5in, top=1.5in, bottom=1.5in]{geometry}

\begin{document}

\section*{Problem 11}

The loud, continued rumbling of thunder after the initial clap is known to anybody who has ever been in a thunderstorm.  (I myself have been in many thunderstorms, especially when backpacking during the monsoon season in the southwest United States.  I have experienced many lightning strikes, many of which were rather close to me---within a few hundred meters.)

This rumbling after the initial clap is not occurring because the lightning bolt itself continues to produce sound.  The discharge of energy from the lightning bolt is instantaneous and does not continue after the initial clap or bang.  The rumbling that occurs after the the first bang is in fact the collective echoes of the initial bang.

Because the thunder-bolt is so loud, these echoes are much louder than a normal echo that would be heard if you shouted as loud as you could.  Both terrain and inhomogeneities within the air can cause these echoes.  Even if there are no irregularities in the terrain that allow the sound to bounce off easily, the flat ground itself can reflect sound.  

Since every single object in the surrounding can reflect sound, the rumble heard by the observer can be interpreted as the accumulation of all these reflections.  There can be variations in the loudness of the rumble---these can be caused by a certain object in the distance that reflects the sound very well.  This object can be a mountain, and the observer will hear it as a sudden increase in volume.

Some people will argue that the rumbling is caused by the size of the thunderbolt itself.  In other words, that the length of the thunderbolt causes the sound from the further regions of the bolt to arrive later.  This is certainly a contribution to the sound, but it is not the main cause of the rumble for these reasons: 1) If this were the only cause of the rumble, then the sound of the thunder would have a sharp dropoff after the last part of the thunderbolt is heard.  This is not the case: rumbling fades away very gradually.  2) If this were true, then rumbles would be louder when the bolt has struck closer to you.  (Assuming the lightning bolt is approximately vertical.)  When the bolt has struck very far from you, all parts of the bolt are approximately equally far from you.  When the bolt has struck very close, the difference in distance is much more significant.  However, this is not the case.  Lightning has more rumbling when it occurs at a moderate distance.  

\begin{remark} I have had the privilege of being in some thunderstorms, especially in areas with a great variety in terrain.  This effect is very noticeable.  When perched atop a ridgeline in a thunderstorm, one can actually hear the ``source" of the rumbling noise travel down valleys.  If the lightning came in the east and there is a valley to the south, one can hear the rumbling noise caused by the lightning travelling from east to west, as if it were flying up the valley.  This is a most remarkable effect that I have witnessed on a few occasions. \end{remark}

There is an additional effect that addresses the "deepness" of the rumble.  Due to Stoke's law of sound attenuation, higher frequency sounds are dissipated into heat at a rate higher than for lower frequency sounds.  (Not related to the inverse square decrease in intensity due to distance.)  Because the rumbles (alternatively, the echoes) long after the initial strike have travelled through much more air prior to its reflection, the higher frequency sounds are dissipated out much more than the lower frequency sounds.  

\vspace{1cm}


\hfill Contestant: Jacob H. Nie



















\end{document}
