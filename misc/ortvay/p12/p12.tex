\documentclass[12pt]{article}
\usepackage{jacob}
\usepackage[left=1.5in,top=1.5in,bottom=1.5in,right=1.5in]{geometry}
\newcommand{\diff}{\,\textit{d}}
\title{}
\date{}
\author{Jacob H. Nie}

\begin{document}

\section*{Problem 12}

Let us work in the reference frame of the center of the Earth.  Technically, this is not inertial, but we were not provided any information about the ratio between the mass of the core to the mass of the earth.  We will treat it as an inertial frame.

There are two sources force the oscillatory force: gravity and the pressure of the surrounding fluid.  Let us first consider gravity. 

We define  $z$ to be the displacement between the center of the core and the center of the Earth.  Note that due to the shell theorem, we are able to ignore everything outside the radius $z+R$ in our considerations of gravity.  This is due to the spherical symmetry it possesses.

\begin{figure}[ht]
	\centering
	\incfig{figure1}
	\caption{This shows everything within the shell of radius $z+R.$}
\end{figure}

Further, note that we can treat the core as a point mass located at its center: since it is spherical and homogeneous, all forces on the core are equivalent to the forces on a point of equal mass at its center.  Thus, our problem is reduced to a point mass inside a sphere of liquid of radius $z+R$ with another empty sphere inside of it, of radius $R$.  

But this is mathematically equivalent to a point mass inside a sphere of liquid of radius $z+R$ with no empty space at all!  That is because if we added liquid into the empty sphere of radius $R$, this liquid would contribute no gravitational force on the point mass due to symmetry.  We can apply shell theorem once more, and the problem is reduced to: a point mass of mass $M_c \equiv (4/3)\pi R^3 \rho_m$ and a sphere of liquid of density $\rho$ and radius $z.$  Hence, the force of gravity on the point mass (and on the core of the original problem) is
\footnote{
	In my first solving, I carried out the full integration, and was dismayed by my stupidity when I finally arrived at the obvious final form.
}


\begin{equation}
	F_g =  \frac{4\pi}{3}GM_c\rho z.
\end{equation}

Let us now examine the force due to the pressures from the surrounding fluid.  To simplify the calculations, we assume the following.  1) Due to the displacement of the core from the center, the force of gravity on an arbitrary volume element of the surrounding liquid will change.  However, we will only analyze how its magnitude changes, and we will ignore all changes in the direction of the gravity force.  2) We assume that at a far enough radius from the center of the Earth $R_0$, the pressures are all the same, a pressure $P_0$.  3) We assume that $z \ll R$.  

First, we calculate the gravitational acceleration at an arbitrary point $(r_0,\phi,\theta)$ in the liquid part of the sphere.  The positive $\hat{\mathbf{k}}$ direction is defined as the dotted line in Figure 2.  $r_0$ is the distance from this arbitrary point to the center of the Earth.  $r$ is the distance from this arbitrary point to the center of the core.  

\begin{figure}[ht]
	\centering
	\incfig{figure2}
	\caption{An arbitrary volume element within the large shell of radius $R_0$.  }
\end{figure}

First, we note that 
\begin{equation}
	r \approx r_0 - z\cos\theta.
\end{equation}
Next, we pretend that the liquid also occupies the space of the core, and that there is simply an `additional' liquid of density $\rho_m - \rho$ in the core.  Thus, the force of gravity on this arbitrary point can be modelled as a force due to two point masses separated by a distance $z.$  Thus,
\begin{equation}
	g = \frac{G}{r_0^2}\left( \rho\cdot \frac{4}{3}\pi r_0^3 \right) + \frac{G}{r^2}\left[ (\rho_m - \rho)\cdot \frac{4}{3}\pi R^3 \right].
\end{equation}
If we combined (2) and (3) and use an approximation, then
\begin{align*}
	g &= \frac{4\pi G}{3} \left[ \rho r_0 + (\rho_m - \rho)R^3 (r_0-z\cos\theta)^{-2} \right] \\
	  &= \frac{4\pi G}{3 r_0^2} \left[ \rho r_0^3 + (\rho_m-\rho)R^3\left( 1+\frac{2z}{r_0}\cos\theta \right) \right].
\end{align*}

These equations allow us to consider the pressure at each point.  

\begin{figure}[ht]
	\centering
	\incfig{figure3}
	\caption{A thin cylindrical volume element.}
\end{figure}

In Figure 3, if we let $A$ be the area of the ends of this cylindrical volume element, then the total force of gravity on the volume element is equal to $P(r_0,\theta) - P_0$ times $A.$  Thus,
\begin{align*}
	P(r_0,\theta) - P_0 &= \int \frac{g\rho}{A} \diff V \\
	P(r_0,\theta) &= \int_{r_0}^{R_0} \diff r_0 \left( \frac{4\pi G \rho}{3} \right)\left[ \rho r_0 + \frac{(\rho_m - \rho)R^3}{r_0^2} + \frac{2(\rho_m - \rho)R^3 z \cos\theta}{r_0^3} \right]\\
		      & \qquad \qquad \qquad + P_0 \\
		      &= P_0 + \frac{4\pi G \rho}{3}\left[ \frac{\rho r_0^2}{2} - \frac{(\rho_m - p)R^3}{r_0} - \frac{(\rho_m - \rho)R^3 z \cos\theta}{r_0^2} \right]_{r_0}^{R_0}
\end{align*}
It is permissible to leave out all constant terms in $P(r_0,\theta)$ that don't vary with $r_0$ and $\theta$.  All these will cancel out when we integrate for the force on the core as a result of the pressure.  So we can write
\[
	P'(r_0,\theta) = -\frac{4\pi G\rho}{3}\left[ \frac{\rho r_0^2}{2} - (\rho_m - \rho)R^3\left( \frac{1}{r_0} + \frac{z\cos\theta}{r_0^2} \right) \right].
\]

To calculate the force that results on the core, we integrate the surface of the sphere.  

\begin{figure}[ht]
	\centering
	\incfig{figure4}
	\caption{The core.  The polar coordinate system is defined such that $\hat{\mathbf{k}}$ is the rightwards direction in this diagram.}
\end{figure}

We integrate with respect to the spherical coordinate system that is centered at the center of the core.  Since $z \ll R$, we can assume $\theta' \approx \theta$, as defined in Figure 4.  The same as previously, $r$ is the distance from the surface element to the center of the core, and $r_0$ is the distance from the surface element to the center of the Earth.

Similarly to previously, $r_0 = r + z\cos\theta = R + z\cos\theta.$

The force on the core is 
 \[
	 F_p = \int_S \diff A \ P(r_0,\theta) \cos\theta = \int_S \diff A \ P'(r_0,\theta') \cos\theta,
\]
which is positive in the $-\hat{\mathbf{k}}$ direction.  Integrating and approximating to first order in $z/R$, this is
\begingroup
\allowdisplaybreaks
\begin{align*}
	F &= \int_0^{\pi} \diff \theta \ R^2 \sin\theta \int_0^{2\pi} \diff \phi \ \left( \frac{4\pi G \rho}{3} \right) \bigg[ -\frac{\rho r_0^2}{2} \\
	  & \qquad \qquad + (\rho_m - \rho) R^3\left( \frac{1}{r_0} + \frac{z\cos \theta}{r_0^2} \right) \bigg] \cos \theta  \\
	  &= \frac{8\pi^2 G \rho R^2}{3} \int_0^{\pi} \diff \theta \ \sin\theta \cos\theta \bigg [ -\frac{\rho (R+z\cos\theta)^2}{2} \\
	  & \qquad \qquad + (\rho_m - \rho) R^3 \left( \frac{1}{R+z\cos\theta} + \frac{z\cos\theta}{(R+z\cos\theta)^2} \right) \bigg]\\
	  &= \frac{8\pi^2 G \rho R^2}{3} \int_0 ^{\pi} \diff \theta \ \sin\theta \cos\theta \bigg [ -\frac{\rho R^2}{2}\left( 1 + \frac{z}{R}\cos\theta \right)^2 \\
	  & \qquad \qquad + (\rho_m - \rho)R^2 \left( 1 + \frac{z}{R}\cos\theta \right)^{-1} \qquad \text{ (equation continued)}\\
	  &\qquad \qquad + (\rho_m - \rho)R^2 \left( \frac{z}{R}\cos\theta \right)\left( 1 + \frac{z}{R}\cos\theta \right)^{-2}\bigg] \\
	  &\simeq \frac{8\pi^2 G\rho R^2}{3} \int_0^{\pi} \diff\theta \ \sin\theta \cos\theta \bigg[- \frac{\rho R^2}{2} - \rho Rz\cos\theta + (\rho_m - \rho)R^2 \\
	  & \qquad \qquad - (\rho_m -\rho)zR\cos\theta + (\rho_m - \rho)zR\cos\theta \bigg] \\
	  &\simeq -\frac{8\pi^2 G \rho^2 R^3 z}{3} \int_0^{\pi} \diff \theta \ \sin\theta \cos^2\theta \\
	  &\simeq -\frac{16\pi^2 G\rho^2 R^3 z}{9}.
\end{align*}
\endgroup

Thus, the total force on the core is 
\begin{align*}
	F &= \left( -\frac{16\pi^2 G \rho^2 R^3}{9} + \frac{4\pi G \rho M_c}{3} \right)z \\
	  &= \frac{16\pi^2 G R^3}{9}(-\rho^2 + \rho \rho_m)z.
\end{align*}
This is positive in the $-\hat{\mathbf{k}}$ direction.  This makes the equation of motion 
\[
	\ddot{z} = -\frac{4\pi}{3}G\rho \left( 1-\frac{\rho}{\rho_m} \right)z.
\]

We can be rather confident in this answer.  It gives the correct behavior of no acceleration at $\rho = \rho_m$. It shows oscillatory behavior if $\rho_m > \rho$.  We are also thankful to learn that the Earth's core is denser than the outer layers.  Otherwise it may shoot out.

\vspace{1cm}

\hfill Contestant: Jacob H. Nie


\end{document}
