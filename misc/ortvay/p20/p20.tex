\documentclass[12pt]{article}
\usepackage{jacob}
\usepackage[left=1.5in,top=1.5in,bottom=1.5in,right=1.5in]{geometry}
\usepackage{braket}

\title{}
\date{}
\author{Jacob H. Nie}

\begin{document}

\section*{Problem 20}
\subsection*{Statement}
A harmonic oscillator is in the $K$-th energy eigenstate.  Then the system is abruptly changed: the frequency of the oscillator is multiplied by a factor $e^{2\gamma}$, where $\gamma$ is an arbitrary real number.  

Express the state vector at the moment of the break and by time $t$ after the break using only the lowering operator and ground state vector of the new system. 

Do the same, but assuming that the initial state of the system was not an energy eigenstate but instead a coherent state characterized by an arbitrary complex number $\beta$.

To know the probability to find the system in the $n$-th energy eigenstate of the new Hamiltonian you have to calculate infinitely many matrix elements.  Construct a generating function with two variables which gives you these matrix elements.  Calculate this probability in the special case $n = K.$ 

Study the limiting case when the parameter $\gamma$ tends to zero.

\subsection*{Summary of Approach}
	First, I establish relations between the raising and lowering operators in the old and new energy eigenstate basis.  I use these relations to establish recursive relations for the sequence $C(n_1,n_0)\equiv \langle \omega_1,n_1|\omega_0,n_0\rangle $.  Then I establish the two-variable generating function 
	\[
		F(x,y) = \sum_{n_1=0}^{\infty}\sum_{n_0=0}^{\infty} C(n_1,n_0)^2 x^{n_1}y^{n_0},
	\] 
which I determine in closed form using techniques for solving partial differential equations.  With this generating function, I can in theory calculate the decomposition of any old energy eigenstate in the new energy eigenbasis, and vice versa.  

The generating function is manipulated in a special way to determine $C(n,n)^2 = |\langle \omega_1,n|\omega_0,n_0\rangle|^2 $ for all $n.$

Then I used some of the established relations to write the initial eigenstate in terms of only the raising operator and the ground state of the new system, as well as after a time $t$.  

I do the same for the case where the oscillator begins in a coherent state.

\subsection*{Solution}

\subsubsection*{Preliminaries}

Some notation: Let $\ket{\omega_0,n_0}$ be the energy eigenstates before $t=0.$ Let $\ket{\omega_1,n_1}$ be the energy eigenstates after $t=0.$ Of course, $\omega_1=\omega_0\exp 2\gamma.$ 

Let the lowering and raising operators in the old basis be $a_0$ and $a_0^{\dagger}$, respectively.  In the new basis, they are $a_1$ and $a_0^{\dagger}$.  Let us now examine these four operators in terms of the position and momentum operators to determine some relations between them.  We have:
\begin{align*}
	a_0 &= \frac{1}{2}\left[\left(\frac{2m\omega_0}{\hbar}\right)^{1/2}X + i\left(\frac{2}{m\omega_0\hbar}\right)^{1/2}P\right] \\
	    &= \left(\frac{1}{2\hbar}\right)^{1/2} \left(\alpha X + \frac{i}{\alpha}P\right) \\
	a_0^{\dagger} &= \left(\frac{1}{2\hbar}\right)^{1/2}\left(\alpha X - \frac{i}{\alpha}P\right) \\
	a_1 &= \left(\frac{1}{2\hbar}\right)^{1/2}\left(\alpha e^{\gamma} X + \frac{i}{\alpha e^{\gamma}}P\right) \\
	a_1^{\dagger} &= \left(\frac{1}{2\hbar}\right)^{1/2}\left(\alpha e^{\gamma} X - \frac{i}{\alpha e^{\gamma}}P\right),
\end{align*}
where we have let $\alpha \equiv  (m\omega_0)^{1/2}$ for convenience.  Substituting using
\begin{align*}
	X &= \left(\frac{\hbar}{2}\right)^{1/2}\left(\frac{1}{\alpha}\right)(a_0+a_0^{\dagger}) \\
	P &= \left(\frac{\hbar}{2}\right)^{1/2}(i\alpha)(a_0^{\dagger}-a_1),
\end{align*}
we get 
\begin{align*}
	a_1 &= (\cosh \gamma) a_0 + (\sinh \gamma) a_0^{\dagger} \\
	a_1^{\dagger} &= (\sinh \gamma) a_0 + (\cosh \gamma) a_0^{\dagger} \\
	a_0 &= (\cosh \gamma) a_1 - (\sinh \gamma) a_1^{\dagger} \\
	a_0^{\dagger} &= -(\sinh \gamma) a_1 + (\cosh \gamma) a_1^{\dagger}.
\end{align*}

The crux of the problem is finding the decomposition
\[
	\ket{\omega_0,n_0} = \sum_{n_1=0}^{\infty} C(n_1,n_0)\ket{\omega_1,n_1},
\] 
where
\[
	C(n_1,n_0) \equiv \langle \omega_1,n_1|\omega_0,n_0\rangle .
\] 

\subsubsection*{A recursive relationship}
Let us lower the state vector $|\omega_0,n_0\rangle $ as follows:
\begin{align*}
	a_0|\omega_0,n_0\rangle &= \sum_{n_1=0}^{\infty} C(n_1,n_0)\left[(\cosh\gamma)a_1-(\sinh\gamma)a_1^{\dagger}\right] |\omega_1,n_1\rangle \\
				&= \sum_{n_1=0}^{\infty} \Big[ C(n_1,n_0)(\cosh\gamma)n_1^{1/2} |\omega_1,n_1-1\rangle \\ 
				& \qquad \qquad - C(n_1,n_0)(\sinh\gamma)(n_1+1)^{1/2}|\omega_1,n_1+1\rangle \Big] \\
				&= \sum_{n_1=1}^{\infty} C(n_1,n_0) (\cosh \gamma)n_1^{1/2} |\omega_1,n_1-1\rangle \\
				& \qquad \qquad - \sum_{n_1=1}^{\infty} C(n_1-1,n_0)(\sinh\gamma)n_1^{1/2}|\omega_1,n_1\rangle  \\
				&= C(1,n_0)(\cosh\gamma)|\omega_1,0\rangle  + \sum_{n_1=1}^{\infty}\Big[ C(n_1+1,n_0)(\cosh\gamma)(n_1+1)^{1/2}\\
				& \qquad \qquad -C(n_1-1,n_0)(\sinh\gamma) n_1^{1/2}\Big]|\omega_1,n\rangle .
\end{align*}
But we also know that
\begin{align*}
	a_0|\omega_0,n_0\rangle &= n_0^{1/2} |\omega_0,n_{0}-1\rangle \\
				&= \sum_{n_1=0}^{\infty} C(n_1,n_0-1)n_0^{1/2}|\omega_1,n_1\rangle .
\end{align*}
From this, we can conclude:
\begin{align}
	C(1,n_0)\cosh\gamma &= C(0,n_0-1)n_0^{1/2} \\
	C(n_1,n_0-1)n_0^{1/2} &= C(n_1+1,n_0) (\cosh \gamma) (n_1+1)^{1/2} \nonumber \\
			      & \qquad - C(n_1-1,n_0)(\sinh \gamma)n_1^{1/2}.
\end{align}
Note that (2) is satisfied even if $n_0=0$, though $C(n_1,-1)$ is not defined. Similarly, it is also satisfied even if $n_1=0$, though $C(-1,n_0)$ is not defined.

It is possible to do the same, but with the raising operator.  This gives:
\begin{align*}
	a_0^{\dagger}|\omega_0,n_0\rangle &= -C(1,n_0)(\sinh \gamma)|\omega_1,0\rangle  + \sum_{n_1=1}^{\infty} \Big[ C(n_1-1,n_0)(\cosh \gamma)n_1^{1/2} \\
					  & \qquad \qquad - C(n_1+1,n_0)(\sinh \gamma) (n_1+1)^{1/2}\Big] |\omega_1,n_1\rangle .
\end{align*}
Of course,
\[
	a_0^{\dagger}|\omega_0,n_0\rangle  = (n_0+1)^{1/2}|\omega_0,n_0+1\rangle ,
\] 
so we arrive at two equations rather similar to (1) and (2):
\begin{align}
	-C(1,n_0)\sinh \gamma &= C(0,n_0+1)(n_0+1)^{1/2} \\
	C(n_1,n_0+1)(n_0+1)^{1/2} &= C(n_1-1,n_0)(\cosh \gamma)n_1^{1/2} \nonumber \\
				  & \qquad - C(n_1+1,n_0)(\sinh \gamma) (n_1+1)^{1/2}.
\end{align}
Note that (4) is still satisfied even if $n_1=0,$ though $C(-1,n_0)$ is not defined.

It can be seen that equations (1), (2), (3), and (4) are sufficient to determine all of $C(n_1,n_0)$ if we know $C(0,0)$ and $C(0,1)$.  This is shown visually in Figure 1.  

\begin{figure}[ht]
	\centering
	\incfig{grid}
	\caption{A diagram showing the points that can be determined recursively given $C(0,0).$  The process can be continued ad infinitum to cover half of all $C(n_1,n_0)$.  The other half is covered with $C(0,1)$.}
\end{figure}

So the logical next step is calculating $C(0,0)$ and $C(0,1)$.  We calculate these explicitly since the integrals are not difficult to do at all.
\begin{align*}
	C(0,0) &= \langle \omega_1,0|\omega_0,0\rangle \\
	       &= \int_{-\infty}^{\infty} \left(\frac{m\omega_1}{\pi \hbar}\right)^{1/4}\exp\left(-\frac{m\omega_1 x^2}{2\hbar}\right)\cdot \left(\frac{m\omega_0}{\pi \hbar}\right)^{1/4}\exp\left(-\frac{m\omega_0 x^2}{2\hbar}\right)\diff x \\
	       &= \left[ \frac{4\omega_0\omega_1}{(\omega_0+\omega_1)^2}\right]^{1/4} \\
	       &= (\cosh \gamma)^{-1/2},
\end{align*}
and
\begin{align*}
	C(1,0) &= \langle \omega_1,1|\omega_0,0\rangle \\
	       &= 0,
\end{align*}
where we have made the astute realization that $\langle \omega_1,1|$ is odd and $|\omega_0,0\rangle $ is even.

With this, we are, in theory, able to calculate all of $C(n_1,n_0).$  (Note that there may be multiple ways of calculating each $C(n_1,n_0).$  We must hope, based on the fact that this describes a physical system, that each way will produce the same result.  Luckily, this appears to be the case when one tries a few examples, which are not shown here.)

First, let us combine the equations into a more inspiring form.  If we combine equations (1) and (3), we find
\begin{equation}
	(n_0+1)C(0,n_0+1)^{2} = n_0(\tanh \gamma)^{2}C(0,n_0-1)^{2}.
\end{equation}
Next, let us square equations (2) and (4): 
\begin{align*}
	C(n_0,n_0-1)^{2}n_0 &= C(n_1+1,n_0)^{2}(\cosh \gamma)^{2}(n_1+1) \\
			    & \quad + C(n_1-1,n_0)^{2}(\sinh \gamma)^{2}n_1 \\
	\quad - 2C(n_1+1,n_0) & C(n_1-1,n_0)(\sinh \gamma \cosh \gamma)n_1^{1/2}(n_1+1)^{1/2}, \\
	C(n_1,n_0+1)^{2}(n_0+1) &= C(n_1+1,n_0)^{2}(\sinh \gamma)^{2}(n_1+1) \\
				& \quad +C(n_1-1,n_0)^{2}n_1 \\
	\quad - 2C(n_1+1,n_0) & C(n_1-1,n_0)(\sinh \gamma \cosh \gamma)n_1^{1/2}(n_1+1)^{1/2}. \\
\end{align*}
Subtracting, 
\begin{equation}
	(n_0+1)C(n_1,n_0+1)^{2}-n_0C(n_1,n_0-1)^{2} = n_1 C(n_1-1,n_0)^{2} - (n_1+1)C(n_1+1,n_0)^{2}.
\end{equation}

\subsubsection*{Determining the generating function}
Now, suppose that $D(n_1,n_0) \equiv C(n_1,n_0)^2$ are the coefficients of a generating function $F.$ More explicitly, 
\[
	F(x,y) = \sum_{n_1=0}^{\infty} \sum_{n_0=0}^{\infty} D(n_1,n_0) x^{n_1} y^{n_0}.
\] 

The following relations hold. (Summation over all $n_0,n_1$ is implied.)
\begin{align*}
	(n_0+1)D(n_1,n_0+1)x^{n_1}y^{n_0} &= \frac{\partial }{\partial y} \left[ D(n_1,n_0+1)x^{n_1} y^{n_0+1} \right] \\
					  &= \frac{\partial F}{\partial y} \\
	-n_0 D(n_1,n_0-1)x^{n_1}y^{n_0} &= -y \frac{\partial }{\partial y} \left[ D(n_1,n_0-1) x^{n_1}y^{n_0-1}\cdot y \right]  \\
					&= -y \frac{\partial (yF)}{\partial y} \\
	n_1 D(n_1-1,n_0) x^{n_1}y^{n_0} &= x\frac{\partial }{\partial x} \left[ D(n_1-1,n_0) x^{n_1-1}y^{n_0}\cdot x \right] \\
					&= x \frac{\partial (xF)}{\partial x} \\
	- (n_1+1) D(n_1+1, n_0) x^{n_1}y^{n_0} &= - \frac{\partial }{\partial x}\left[ D(n_1+1,n_0) x^{n_1+1} y^{n_0} \right] \\
					       &= -\frac{\partial F}{\partial x}.
\end{align*}
(Note that there are some weird situations when either $n_0=0$ or $n_1=0$. Because (6) is still satisfied in these cases, there is no issue.)

This leads us to the differential equation
\begin{align}
	\frac{\partial F}{\partial y} - y\frac{\partial (yF)}{\partial y} &= x\frac{\partial (xF)}{\partial x} - \frac{\partial F}{\partial x} \nonumber \\
	\implies \qquad \qquad (x+y) F &= \frac{\partial F}{\partial y}(1-y^2) + \frac{\partial F}{\partial x}(1-x^2).
\end{align}

Working from (5), we can also establish the following relations:
\begin{align*}
	(n_0+1)D(0,n_0+1)y^{n_0} &= \frac{\partial }{\partial y} \left[ D(0,n_0+1) y^{n_0+1} \right] \\
				 &= \frac{\partial F(0,y)}{\partial y} \\
	(\tanh \gamma)^2 n_0 D(0,n_0-1) y^{n_0} &= (\tanh \gamma)^2 y \frac{\partial }{\partial y} \left[ D(0,n_0-1) y^{n_0-1}\cdot y \right] \\
						&= (\tanh \gamma)^{2} y \frac{\partial}{\partial y}\left[ y F(0,y) \right] .
\end{align*}
Using the equality in (5), we can construct the differential equation
\[
	\frac{\partial F(0,y)}{\partial y}\left[ 1- y^2 (\tanh \gamma)^{2} \right] = (\tanh \gamma)^{2} y F(0,y).
\] 
This can be integrated, giving
\[
	F(0,y) = \frac{c}{\left[ 1-(\tanh \gamma)^2 y^2 \right] ^{1/2}},
\] 
where $c$ is a constant of integration.  Since we know $F(0,0) = C(0,0)^{2} = (\cosh \gamma)^{-1},$ we can narrow this down to
\begin{equation}
	F(0,y) = \frac{1}{\left[ \cosh^2 \gamma - (\sinh \gamma)^{2}y^{2} \right] ^{1/2}}.
\end{equation}
(As expected, $F(0,1) = \sum_{n_0} D(0,n_0) = \sum_{n_0} |\langle \omega_1,0|\omega_0,n_0\rangle|^{2}  = 1.$)

With this boundary condition, we are prepared to tackle (7).  Let us first find a function $h(x,y)$ that satisfies (7), but probably won't satisfy (8).  Assume that $h(x,y)=j(x)k(y)$, i.e. that it is separable.  (7) becomes
\[
	(x+y) j(x) k(y) = \frac{\partial }{\partial y}(\log k)(1-y^{2}) + \frac{\partial }{\partial x}(\log j)(1-x^2).
\] 
We can integrate for $j(x)$ and $k(y)$ separately, and we find that one possible solution for $h(x,y)$ is
\begin{equation}
	h(x,y) = \left[ (1-x^2) (1-y^2) \right] ^{-1/2}.
\end{equation}

Next, let us find a function $u(x,y)$ such that
\[
	(1-x^2)\frac{\partial u}{\partial x} + (1-y^2)\frac{\partial u}{\partial y} = 0.
\] 
We guess that $u(x,y) = \varphi(p)$, where $p$ is a function of $x$ and $y$.  It can be verified that 
\begin{equation}
	p (x,y) = \frac{(x+1)(y-1)}{(x-1)(y+1)},
\end{equation}
in order to satisfy the differential equation.  We can now put everything together: let us now solve (7) with an ansatz $F(x,y) = h(x,y) \varphi(p)$.  This gives
\begin{align*}
	0 &= (y^2 - 1)\frac{\partial F}{\partial y} + (x^2 - 1)\frac{\partial F}{\partial x} + (x+ y) F \\
	  &= (y^2 - 1)\frac{\partial h}{\partial y}\varphi(p) + h(x,y) \frac{\diff \varphi}{\diff p} \cdot \frac{2(y-1)(x+1)}{(x-1)(y+1)} \\
	  &   \qquad  + (x^2 - 1)\frac{\partial h}{\partial x}\varphi(p) + h(x,y) \frac{\diff \varphi}{\diff p} \cdot \frac{(-2)(x+1)(y-1)}{(x-1)(y+1)} \\
	  & \qquad + (x+y)h(x,y)\varphi(p)  \\
	  &= \varphi(p)\left[ (y^2 -1 )\frac{\partial h}{\partial y} + (x^2 - 1)\frac{\partial h}{\partial x} + (x+y) h \right] .
\end{align*}
Of course, our $h(x,y)$ already satisfies this.  It only remains to solve for $\varphi(p)$, which we can do with use of (8).  
\begin{align*}
	F(0,y) &= h(0,y) \varphi[p(0,y)] \\
	       &= \frac{1}{(1-y^2)^{1/2}} \cdot \varphi \left[ \frac{1-y}{1+y} \right] \\
	       &= \frac{(\cosh \gamma)^{-1}}{\left[ (1-y\tanh \gamma)(1+y\tanh \gamma) \right] ^{1/2}}.
\end{align*}
It is profitable to rewrite $y$ in terms of $p(0,y)$, which we will abbreviate as $p$ for now, for the sake of convenience.  It can be verified that
\[
y = \frac{1-p}{1+p}.
\] 
Thus,
\begin{align*}
	\varphi (p) &= \frac{1}{\cosh \gamma} \left[ \left( \frac{1 - y\tanh \gamma}{1 - y} \right) \left( \frac{1 + y\tanh \gamma}{1+ y} \right)  \right] ^{-1/2} \\
		    &= \frac{1}{\cosh \gamma} \left[ \frac{(p+1)^2 - (\tanh \gamma)^{2} (p-1)^2}{4p} \right] ^{-1/2}.
\end{align*}
This was $\varphi[p(0,y)],$ but $\varphi[p(x,y)]$ must be the same.  We can then substitute (10) into our equation for $\varphi(p)$ to find that
\begin{equation}
	\varphi[p(x,y)] = \frac{\left[ (x^2 -1)(y^2-1) \right] ^{1/2}\text{sech}\ \gamma}{\left[ (xy-1)^2 - (x-y)^2\tanh^2\gamma \right] ^{1/2}}.	
\end{equation}
Finally, we can combine (9) and (11) to find that
\begin{equation}
	F(x,y) = \left[ (xy-1)^2 \cosh^2 \gamma - (x-y)^2 \sinh^2 \gamma \right] ^{-1/2}.
\end{equation}
It can be verified that this satisfies both (7) and (8).  This is our desired generating function for the sequence $D(n_1,n_0)$.  Each of $D(n_1,n_0)$ can be determined with the following equation:
\begin{equation}
	D(n_1,n_0) = \frac{1}{n_1!n_0!} \frac{\partial^{n_1+n_0} F}{\partial x^{n_1}\partial y^{n_0}}\bigg|_{x=0,y=0},
\end{equation}
with $F(x,y)$ defined in (12). The first 100 values of $D$ are shown in Table 1.  (Some calculations will reveal that these are in fact identical to the values of $C(n_1,n_0)$ that were previously computed recursively.)
\begin{table}[ht]
\centering
\begin{tabular}{|c|c|c|c|c|c|c|c|c|c|}
\hline
0.65 & 0 & 0.19 & 0 & 0.08 & 0 & 0.04 & 0 & 0.02 & 0 \\ 
\hline
0 & 0.27 & 0 & 0.24 &    0 & 0.17 & 0 & 0.12 & 0 & 0.08 \\ 
\hline
0.19 & 0 & 0.01 & 0 & 0.09 & 0 & 0.13 & 0 &    0.13 & 0 \\ 
\hline
0 & 0.24 & 0 & 0.06 & 0 & 0.0002 & 0 & 0.02 & 0 &    0.06 \\ 
\hline
0.08 & 0 & 0.09 & 0 & 0.12 & 0 & 0.06 & 0 & 0.01 & 0 \\ 
\hline
0 & 0.17 &    0 & 0.0002 & 0 & 0.05 & 0 & 0.08 & 0 & 0.07 \\ 
\hline
0.04 & 0 & 0.13 & 0 &    0.06 & 0 & 0.001 & 0 & 0.02 & 0 \\ 
\hline
0 & 0.12 & 0 & 0.02 & 0 & 0.08 & 0 &    0.05 & 0 & 0.007 \\ 
\hline
0.02 & 0 & 0.13 & 0 & 0.01 & 0 & 0.02 & 0 & 0.06 &    0 \\ 
\hline
0 & 0.08 & 0 & 0.06 & 0 & 0.07 & 0 & 0.007 & 0 & 0.01 \\
\hline
\end{tabular}
\caption{The first 100 values of $D$, where $\gamma=1.$ $D(0,0)$ is at the top left corner. Calculations done in Mathematica.}
\end{table}

\subsubsection*{State in terms of only the raising operator and new ground state}

We will decompose the ground state in the old system in the new system's eigenbasis.  Then we will use the old raising operator (which we can relate the new raising operator to) to raise the ground state into the $K$-th eigenstate. 

We do not have to use the full generating function.  One recursive relation will suffice.  Let us apply (2) with the conditions $(n_1, 0)$.  
\begin{align*}
	C(n_1-1,0)(\sinh \gamma)n_1^{1/2} &= C(n_1+1,0)(\cosh \gamma)(n_1+1)^{1/2} \\
	C(n_1+1,0) &= \left( \frac{n_1}{n_1+1} \right)^{1/2} (\tanh \gamma) C(n_1-1,0).
\end{align*}
We are familiar with the two ``initial conditions'' $C(0,0) = (\cosh\gamma)^{-1/2}$ and $C(1,0) = 0$.  Thus, we can easily find
\begin{equation}
	C(n_1,0) = \left[ \frac{(n_1-1)!!}{n_1!!} \right]^{1/2} (\tanh \gamma)^{n_1/2} (\cosh \gamma)^{-1/2},
\end{equation}
which is true only for even $n_1$.  ($n_1!!$ is the double factorial $n_1(n_1-2)\cdots$.)  Thus,
\begin{align}
	|\omega_0,0\rangle &= \sum_{n_1=0}^{\infty} \langle \omega_1,n_1|\omega_0,0\rangle |\omega_1,n_1\rangle \nonumber  \\
			   &= \sum_{n = 0}^{\infty} \left[ \frac{(2n-1)!!}{(2n)!!} \right]^{1/2} (\tanh \gamma)^n (\cosh \gamma)^{-1/2} \frac{(a_1^{\dagger})^{2n}}{(2n!)^{1/2}} |\omega_1,0\rangle . 
\end{align}
Of course,
\begin{align}
	|\omega_0, K\rangle &= \frac{(a_0^{\dagger})^{K}}{(K!)^{1/2}}|\omega_0,0\rangle \\
			    &= \frac{(-a_1\sinh\gamma + a_1^{\dagger}\cosh \gamma)^{K}}{(K!)^{1/2}}\left\{ \qquad \qquad \right\}, 
\end{align}
where (15) should be inserted into the brackets in (16).

The time evolution is simply:
\begin{align*}
	|\psi(t)\rangle  &= \exp\left( -\frac{iHt}{\hbar} \right) |\omega_0, K\rangle \\
			 &= \exp \left[ -i\omega_1 t \left(a_1^{\dagger}a_1 + \frac{1}{2}\right) \right] |\omega_0,K\rangle .
\end{align*}

\subsubsection*{Coherent State}
The coherent state is defined by
\[
	a_0|\beta\rangle  = \beta |\beta\rangle .
\] 
Suppose that the coherent state's decomposition in the old energy eigenbasis is $\sum A_n |\omega_0,n\rangle $.  Then:
\begin{align*}
	\sum_{n=1}^{\infty} A_n n^{1/2} |n-1\rangle &= \sum_{n=0}^{\infty} A_n \beta |n\rangle \\
	A_{n+1} &= \frac{\beta}{(n+1)^{1/2}} A_n \\
	A_n &= \frac{\beta^{n}}{(n!)^{1/2}} A_0.
\end{align*}
The state needs to be normalized:
\begin{align*}
	1 &= \sum_{n=0}^{\infty} \frac{|\beta|^{2n}}{n!}|A_0|^2  \\
	A_0 &= \exp\left( -\frac{|\beta|^2}{2} \right).
\end{align*}
Thus,
\begin{equation}
	|\beta\rangle  = \sum_{n=0}^{\infty} \frac{\beta^n}{(n!)^{1/2}}\exp\left( -\frac{|\beta|^2}{2} \right) |\omega_0,n\rangle .
\end{equation}
Simply plugging in (17) with $n = K$ will yield the coherent state in terms of only the raising operator and the ground state in the new system.  It can also easily be evolved by using the same propagator as in the previous section. 


\subsubsection*{Probabilities for $n = K$ }
We seek the probabilities $D(n,n) = |\langle \omega_1,n|\omega_0,n\rangle |^2$.  Although a method for calculating these probabilities is already given in equation 13, we seek a simpler form.  Indeed, a simpler form is strongly suggested when plotting these values $D(n,n)$, as is done in Figure 2. 

\begin{figure}[ht]
	\centering
	\incfig{plot7}
	\caption{The values $D(n,n)$ for $n \in \{0, \ldots, 20\}$, and $\gamma =0.4$.  (Calculations done in Mathematica.)}
\end{figure}

The behavior seems to be similar to that of the sinc function.  However, the origins of this behavior is unknown to me, despite my efforts.

\vspace{1cm}

\hfill Contestant: Jacob H. Nie






























\end{document}
