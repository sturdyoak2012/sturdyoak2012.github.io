\documentclass[12pt]{article}
\usepackage{jacob}
\usepackage[left=1.5in,top=1.5in,bottom=1.5in,right=1.5in]{geometry}

\newcommand{\diff}{\,\textit{d}}
\title{}
\date{}
\author{Jacob H. Nie}

\begin{document}

\section*{Problem 7}

Some notation and conventions used:
\begin{align*}
	\mathcal{F}[x(t)]  &= \left(\frac{1}{2\pi}\right)^{1/2} \int_{-\infty}^{\infty} x(t) \exp(-i\omega t)\diff t \\
		    &= \hat{x}(\omega) \\
	\mathcal{F}^{-1}[\hat{x}(\omega)] &= \left( \frac{1}{2\pi} \right)^{1/2} \int_{-\infty}^{\infty} \hat{x}(\omega) \exp(i\omega t)\diff t \\
					  &= x(t)
\end{align*}

\begin{theorem}
	If \[
		x(t) = \int_{-\infty}^{\infty} G(t-t')F(t') \diff t',
	\] 
	then \[
		\hat{x}(\omega) = (2\pi)^{1/2}\hat{G}(\omega) \hat{F}(\omega).	
	\] 

\end{theorem}

\begin{proof}
	\begin{align*}
		\hat{x}(\omega)  &= \left(\frac{1}{2\pi}\right)^{1/2} \int_{-\infty}^{\infty} x(t) \exp(-i\omega t)\diff t \\
				 &= \left(\frac{1}{2\pi}\right)^{1/2} \int_{-\infty}^{\infty} \diff t \exp\left[\left(-i\omega(t-t')\right)\right] G(t-t') \int_{-\infty}^{\infty} \diff t' \exp(-i\omega t') F(t') \\ 
				 &= \left(\frac{1}{2\pi}\right)^{1/2} \int_{-\infty}^{\infty} \diff (t-t') \exp\left[\left(-i\omega(t-t')\right)\right] G(t-t') \int_{-\infty}^{\infty} \diff t' \exp(-i\omega t') F(t') \\
				 &= \left(\frac{1}{2\pi}\right)^{1/2} (2\pi)^{1/2} \hat{G}(\omega)\cdot (2\pi)^{1/2} \hat{F}(\omega) \\
				 &= (2\pi)^{1/2}\hat{G}(\omega) \hat{F}(\omega).
	\end{align*}
\end{proof}

Let us calculate $\hat{F}(\omega).$  This is:
\begin{align*}
	\mathcal{F}[m\ddot{x}(t)] &= \left(\frac{1}{2\pi}\right)^{1/2} m \int_{-\infty}^{\infty} \frac{\diff^{2}x}{\diff t^{2}} \exp(-i\omega t)\diff t \\
				  &= \left(\frac{1}{2\pi}\right)^{1/2}m\left[\dot{x}\exp(-i\omega t)\Big|_{-\infty}^{\infty} + i\omega x \exp(-i\omega t)\Big|_{-\infty}^{\infty} - \omega^2 \int_{\infty}^{\infty} x \exp(-i\omega t)\diff t \right] \\
				  &= -\omega ^2 m \hat{x}(\omega),
\end{align*}
where we have assumed that $x$ and $\dot{x}$ tend towards 0 as t goes to $\pm \infty$.

This allows us to calculate $\hat{G}(\omega) :$
\begin{align*}
	\hat{G}(\omega) &= \left(\frac{1}{2\pi}\right)^{1/2}\frac{\hat{x}(\omega)}{\hat{F}(\omega)} \\
			&= -\left(\frac{1}{2\pi}\right)^{1/2} \left(\frac{1}{\omega^2 m}\right).
\end{align*}
Thus, 
\[
	G(t) = -\left( \frac{1}{2\pi m^2} \right)^{1/2} \mathcal{F}^{-1}\left[ \omega^{-2} \right]
\] 

Let us introduce the following theorem: 
\begin{theorem}
	\[
		\mathcal{F}^{-1}\left[ \frac{\diff }{\diff \omega}\hat{f}(\omega) \right] = -i t f(t).
	\]
	
\end{theorem}
\begin{proof}
	\begin{align*}
		\mathcal{F}^{-1}\left[ \frac{\diff}{\diff \omega}\hat{f}(\omega) \right] &= \left( \frac{1}{2\pi} \right)^{1/2} \int_{-\infty}^{\infty} \diff \omega \ \exp(i\omega t)\frac{\diff }{\diff \omega}\hat{f}(\omega) \\
											 &= \left( \frac{1}{2\pi} \right)\int_{-\infty}^{\infty} \diff \omega \ \exp(i\omega t) \frac{\diff }{\diff \omega}\int_{-\infty}^{\infty} \diff t' \ \exp(-i\omega t')f(t') \\
											 &= \left( \frac{-i}{2\pi} \right)\int_{-\infty} ^{\infty} \diff t' \ t' f(t') \int_{-\infty}^{\infty} \diff \omega \ \exp \left[ i \omega(t-t') \right] \\
											&=  -i\int_{-\infty}^{\infty} \diff t' \ t'f(t')\delta(t-t') \\
											 &= -itf(t).
	\end{align*}
\end{proof}

Now let us also introduce the transform of the sgn function. 
\begin{theorem}
	\[
		\mathcal{F}[\emph{sgn}(t)] = \left( \frac{1}{2\pi} \right)^{1/2} \left( \frac{2}{i\omega} \right).
	\]
	
\end{theorem}
\begin{proof}
	We have the integral
	\[
		\mathcal{F} [ \text{sgn}(t) ] = \left( \frac{1}{2\pi} \right)^{1/2} \int_{-\infty}^{\infty} \text{sgn}(t)\exp(-i\omega t)\ \diff t. 
	\]
	It is also known that
	\[
		\frac{\diff}{\diff t}\text{sgn}(t) = 2\delta(0).
	\]
	We can use this fact to integrate by parts as follows:
	\begin{align*}
		\mathcal{F}[\text{sgn}(t)] &= \left( \frac{1}{2\pi} \right)^{1/2} \bigg[ \frac{i}{\omega} \text{sgn}(t) \exp(-i\omega t)\big|_{-\infty}^{\infty} - \frac{2i}{\omega} \int_{-\infty}^{\infty} \delta(0) \exp(-i \omega t)\ \diff t \bigg] \\
					   &= -\left( \frac{1}{2\pi} \right)^{1/2} \left( \frac{2i}{\omega} \right).
	\end{align*}
	
\end{proof}

We can combine these two theorems:
\begin{align*}
	\mathcal{F}^{-1}[\omega^{-2}] &= \mathcal{F}^{-1} \left\{ \frac{\diff}{\diff \omega} \mathcal{F}\left[ -(2\pi)^{1/2} \left( \frac{i}{2} \right)\text{sgn}(t) \right] \right\} \\
				      &= i (2\pi)^{1/2} \left( \frac{i}{2} \right) t \ \text{sgn}(t) \\
				      &= -\left( 2\pi\right)^{1/2} \cdot \frac{t}{2} \ \text{sgn}(t).
\end{align*}
Thus,
\begin{align*}
	G(t) &= \left( \frac{1}{2\pi} \right)^{1/2} \left( \frac{1}{m} \right) \left( 2\pi \right)^{1/2} \left( \frac{t}{2} \right)\text{sgn}(t) \\
	     &= \frac{|t|}{2m}.
\end{align*}






\vspace{1cm}

\hfill Contestant: Jacob H. Nie

\end{document}



























